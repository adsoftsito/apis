% Author: Adolfo Centeno 
% Waves Lab

 
\documentclass{beamer}
\setbeamertemplate{navigation symbols}{}
\usepackage[utf8]{inputenc}
\usepackage{beamerthemeshadow}
\usepackage{listings}
\usepackage{hyperref}

\hypersetup{
  colorlinks=true,
  linkcolor=blue!50!red,
  urlcolor=green!70!black
}

\begin{document}
\title{ITESM}  
\subtitle{Campus Puebla\\ADMINISTRACION DE PROYECTOS DE INGENIERIA DE SOFTWARE
}
\author{Adolfo Centeno}
\date{\today} 


\begin{frame}
\titlepage
\end{frame}

\begin{frame}\frametitle{REDIS - Remote Dictionary Server}
\tableofcontents
\end{frame} 


\section{W11 - Sesion }

\begin{frame}

    
\textbf{W11 - Actividades:}


\begin{enumerate}

\item
    Resolver dudas de la integracion del proyecto de mapas con REST API
\item
    Publicar fronted y rest-api en dockers de acuerdo a puertos asignados en excel
\item
   Instalar   
    \href{https://redis.io/topics/quickstart}{Redis}
    
\item
    \href{https://codeburst.io/redis-what-and-why-d52b6829813}{Redis - Comandos basicos}
\item
    \href{https://redis.io/topics/data-types-intro}{Redis - Tipos de datos}
    
\item
   Laboratorio   
    \href{https://medium.com/@kumarshivam_66534/implementation-of-spring-boot-data-redis-for-caching-in-my-application-218d02c31191}{Springboot con Redis}
	 
\end{enumerate}

	


\end{frame}


\section{W11  - Tareas }

\begin{frame}


\textbf{W11  - Tareas}


\begin{enumerate}

\item
	Probar la API de springboot con redis desde swagger (subir SS de llamadas) y curl (subir SS de llamadas desde shell)
	
\end{enumerate} 


\end{frame}




\end{document}
