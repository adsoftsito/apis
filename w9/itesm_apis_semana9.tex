% Author: Adolfo Centeno 
% Waves Lab

 
\documentclass{beamer}
\setbeamertemplate{navigation symbols}{}
\usepackage[utf8]{inputenc}
\usepackage{beamerthemeshadow}
\usepackage{listings}
\usepackage{hyperref}

\hypersetup{
  colorlinks=true,
  linkcolor=blue!50!red,
  urlcolor=green!70!black
}

\begin{document}
\title{ITESM}  
\subtitle{Campus Puebla\\ADMINISTRACION DE PROYECTOS DE INGENIERIA DE SOFTWARE
}
\author{Adolfo Centeno}
\date{\today} 


\begin{frame}
\titlepage
\end{frame}

\begin{frame}\frametitle{Java SpringBoot - Security, JsonWebToken}
\tableofcontents
\end{frame} 


\section{W9 - Sesion }

\begin{frame}

    
\textbf{W9 - Actividades:}


\begin{block}{clone spring template with Security dependency}


 
    \emph{git clone https://github.com/adsoftsito/springboot-jwt.git}
\end{block}

\begin{enumerate}

\item check jwt dependency  \href{https://github.com/adsoftsito/ng5-api/blob/master/Dockerfile}{pom.xml}

\item ssh -i user user@104.198.244.0    
       
\item sudo -u postgres psql

\item
	check Run and Check Results in theJsonWebToken Lab: \href{https://grokonez.com/spring-framework/spring-boot/spring-security-jwt-authentication-postgresql-restapis-springboot-spring-mvc-spring-jpa}{JsonWebToken Lab}.	
\item
    Add  \href{https://github.com/adsoftsito/maps-api-jwt/blob/master/src/main/java/com/grokonez/jwtauthentication/model/Estado.java}{model}, \href{https://github.com/adsoftsito/maps-api-jwt/blob/master/src/main/java/com/grokonez/jwtauthentication/controller/EstadoController.java}{controller}, \href{https://github.com/adsoftsito/maps-api-jwt/blob/master/src/main/java/com/grokonez/jwtauthentication/repository/EstadoRepository.java}{repository} for estados
    
\item
    Add  model, controller, repository for municipios
\item
    Add  model, controller, repository for categorias
\item
    Add  model, controller, repository for empresas
\item
    Populate database from csv
\item
    Test api with \href{https://www.postman.com/downloads/}{Postman}.	
\item
	Run as Docker \href{https://github.com/adsoftsito/ng5-api/blob/master/Dockerfile}{Dockerfile}
	 
\end{enumerate}

	


\end{frame}


\section{W9  - Tareas }

\begin{frame}


\textbf{W9  - Tareas}


\begin{enumerate}
\item
	Publicar como Docker en servidor de pruebas con el puerto asignado, conectar \href{https://github.com/adsoftsito/bigdata-maps}{frontend de mapas} con rest-api (Nota: por el momento correr frontend en localhost)
	
\end{enumerate} 


\end{frame}




\end{document}
