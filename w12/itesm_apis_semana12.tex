% Author: Adolfo Centeno 
% Waves Lab

 
\documentclass{beamer}
\setbeamertemplate{navigation symbols}{}
\usepackage[utf8]{inputenc}
\usepackage{beamerthemeshadow}
\usepackage{listings}
\usepackage{hyperref}

\hypersetup{
  colorlinks=true,
  linkcolor=blue!50!red,
  urlcolor=green!70!black
}

\begin{document}
\title{ITESM}  
\subtitle{Campus Puebla\\ADMINISTRACION DE PROYECTOS DE INGENIERIA DE SOFTWARE
}
\author{Adolfo Centeno}
\date{\today} 


\begin{frame}
\titlepage
\end{frame}

\begin{frame}\frametitle{REDIS - Nodejs and Python}
\tableofcontents
\end{frame} 


\section{W12 - Sesion }

\begin{frame}

    
\textbf{W11 - Actividades:}


\begin{enumerate}

\item
    Resolver dudas de las tareas pendientes 
\item
	Revisar rubrica de  \href{https://redis.io/topics/quickstart}{parcial 2}

\item
   Laboratorio REDIS    
    \href{https://redis.io/topics/quickstart}{Nodejs}

\item
   Laboratorio REDIS    
    \href{https://redis.io/topics/quickstart}{Python}
	 
\end{enumerate}

	


\end{frame}


\section{W11  - Tareas }

\begin{frame}


\textbf{W12  - Tareas}


\begin{enumerate}

\item
  Subir SS de labs Nodejs y Python con REDIS
  	
\end{enumerate} 


\end{frame}




\end{document}
